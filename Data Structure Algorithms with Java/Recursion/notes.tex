Recursion

1} is a programming technique in which a function calls itself in order to solve a problem. It typically involves a base case that stops the recursion and a recursive case that breaks the problem into smaller subproblems.
     -Iteratiion -> Repetition of a block of code until a certain condition is met.
     -Recursion -> A function that calls itself to solve a problem.
     -loop -> A control structure that allows for repeated execution of a block of code.

  
✅ more uses -> sorting algorithms, backtracking, dynamic programming, and divide-and-conquer algorithms. tree traversal, graph traversal, and mathematical problems like factorials and Fibonacci numbers.


✅   Advantages of Recursion:
      -Simplicity: Recursive code can be easier to read and understand, especially for problems that have a natural recursive structure.
      -Modularity: Recursive functions can be modular and reusable, allowing for cleaner code.
      -Expressiveness: Some problems are more naturally expressed using recursion, making the code more intuitive.

✅ Disadvantages of Recursion:
      -Performance: Recursive functions can be less efficient than iterative solutions due to the overhead of function calls and potential for stack overflow if the recursion depth is too great. 
          -Memory Usage: Recursive functions can consume more memory due to the call stack, especially if the recursion depth is large.

